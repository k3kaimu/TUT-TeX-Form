\documentclass[11pt,dvipdfmx]{jsarticle}
\usepackage{pdffill}
\usepackage{amssymb}

\usepackage{pythontex}

\begin{pycode}
import mojimoji
import yaml
import sys
import importlib

pinfo = yaml.load(open("../pinfo.yaml", "r+", encoding='utf-8'))
\end{pycode}

\newcommand{\ZenToHan}[1]{mojimoji.zen_to_han(u'#1')}
\newcommand{\HanToZen}[1]{mojimoji.han_to_zen(u'#1')}


\def\Date{平成XX年XX月XX日}
\def\ScholarshipID{6XX06XXXXXX}
\def\TitleJP{研究課題題目}
\def\ResearchAbstract{
概要はこんな感じで書いていきます.
概要はこんな感じで書いていきます.
概要はこんな感じで書いていきます.
概要はこんな感じで書いていきます.
概要はこんな感じで書いていきます.
概要はこんな感じで書いていきます.
概要はこんな感じで書いていきます.
概要はこんな感じで書いていきます.
概要はこんな感じで書いていきます.
概要はこんな感じで書いていきます.
}
\def\IHaveAchievement1{〇}
\def\IHaveAchievement2{}
\def\IHaveAchievement3{}
\def\IHaveAchievement4{〇}
\def\IHaveAchievement5{〇}
\def\IHaveAchievement6{〇}
\def\IHaveAchievement7{〇}
\def\IHaveAchievement8{〇}
\def\IHaveAchievement9{}
\def\IHaveAchievement10{}
\def\AchievementsAbstract{
業績の要旨はこんな感じで書いていきます.
業績の要旨はこんな感じで書いていきます.
業績の要旨はこんな感じで書いていきます.
業績の要旨はこんな感じで書いていきます.
業績の要旨はこんな感じで書いていきます.
業績の要旨はこんな感じで書いていきます.
業績の要旨はこんな感じで書いていきます.
業績の要旨はこんな感じで書いていきます.
}
\def\IHaveSubmittedPledge{\checkmark}
\def\IHaveNotSubmittedPledge{}
\def\IWillSubmitPledgeOn{平成  年  月}
\def\IHaveDoneProcedure{\checkmark}
\def\IHaveNotDoneProcedure{}
\def\IWillDoProcedure{}
\def\IWillDoProcedureOn{平成  年  月}
\def\ReasonForRecommendation{
申請者である~君は優秀で~~.
って感じでかいていきます.よろしくお願いいたします。
申請者である~君は優秀で~~.
って感じでかいていきます.よろしくお願いいたします。
申請者である~君は優秀で~~.
って感じでかいていきます.よろしくお願いいたします。
}


\begin{document}
% 下の一行をコメントアウトするとグリッド等が無くなる
\pfdefaultoption{grid,draft}

% 1ページ目
\pdffill[page=1]{pdf/29henkanshinsei}{
\fontsize{11pt}{7mm}\selectfont

% 日付
\pfnode[left] (19.75, 27.45) {\Date}

% 大学院名
\pfnode[right, text width=14.6cm, text centered] (5.2, 22.1) {\py{pinfo["大学"]}大学院}

% 研究科名
\pfnode[right, text width=6.7cm, text centered] (5.2, 19.8) {工学研究科 \py{pinfo["所属"]}}

% 学籍番号
\pfnode[right, text width=4.9cm, text centered] (14.9, 19.8) {\py{mojimoji.han_to_zen(pinfo["学籍番号"])}}


% 奨学生番号
\pfnode[right] (5.0, 18.7) {
\edef\xArgs{{6.1mm}{\ScholarshipID}}
\expandafter\splitboxes\xArgs
}

% 生年月日
\pfnode[left] (19.6, 18.7) {\py{'{0}年{1}月{2}日'.format(int(pinfo["生年月日"][0])-1988, pinfo["生年月日"][1], pinfo["生年月日"][2])}}

% 郵便番号
\pfnode[right] (5.6, 17.8) {\py{mojimoji.han_to_zen(pinfo["郵便番号"])}}

% 電話番号
\pfnode[right] (13.6, 17.8) {\py{mojimoji.han_to_zen(pinfo["電話番号"])}}

% 住所
\pfnode[right] (5.3, 17.0) {\py{pinfo["住所"]}}

% 題目
\pfnode[right] (2.9, 15.0) {\TitleJP}


% 概要
\pfnode[below right, text width=16.7cm, align=justify] (2.9, 14.4) {\ResearchAbstract}


% 教育研究活動等の業績
\pfnode[right] (1.6, 6.7) {\fontsize{32pt}{0}\selectfont\IHaveAchievement1}
\pfnode[right] (7.7, 6.7) {\fontsize{32pt}{0}\selectfont\IHaveAchievement2}
\pfnode[right] (13.8, 6.7) {\fontsize{32pt}{0}\selectfont\IHaveAchievement3}
\pfnode[right] (1.6, 5.4) {\fontsize{32pt}{0}\selectfont\IHaveAchievement4}
\pfnode[right] (7.7, 5.4) {\fontsize{32pt}{0}\selectfont\IHaveAchievement5}
\pfnode[right] (13.8, 5.4) {\fontsize{32pt}{0}\selectfont\IHaveAchievement6}
\pfnode[right] (1.6, 4.1) {\fontsize{32pt}{0}\selectfont\IHaveAchievement7}
\pfnode[right] (7.7, 4.1) {\fontsize{32pt}{0}\selectfont\IHaveAchievement8}
\pfnode[right] (13.8, 4.1) {\fontsize{32pt}{0}\selectfont\IHaveAchievement9}
\pfnode[right] (1.6, 2.8) {\fontsize{32pt}{0}\selectfont\IHaveAchievement10}
}



% 2ページ目
\pdffill[page=2]{pdf/29henkanshinsei}{
\fontsize{11pt}{6.22mm}\selectfont
\parindent=11pt
% 特に優れた業績の要旨
\pfnode[below right, text width=18cm, align=justify] (1.7, 27.33) {\AchievementsAbstract}


\pfnode[right] (3.4, 9.85) {\fontsize{15pt}{0}\selectfont\IHaveSubmittedPledge}
\pfnode[right] (7.65, 9.85) {\fontsize{15pt}{0}\selectfont\IHaveNotSubmittedPledge}
\pfnode[right] (8.5, 9.85) {提出予定({\IWillSubmitPledgeOn} 大学へ提出予定)}

\pfnode[right] (3.4, 7.55) {\fontsize{15pt}{0}\selectfont\IHaveDoneProcedure}
\pfnode[right] (7.65, 7.55) {\fontsize{15pt}{0}\selectfont\IHaveNotDoneProcedure}
\pfnode[right] (11.35, 7.55) {\fontsize{15pt}{0}\selectfont\IWillDoProcedure}
\pfnode[right] (12.2, 7.55) {手続き予定({\IWillDoProcedureOn})}

}


% 3ページ目
\pdffill[page=3]{pdf/29henkanshinsei}{
% 奨学生番号
\pfnode[right] (3.85, 27.0) {
\edef\xArgs{{6.1mm}{\ScholarshipID}}
\expandafter\splitboxes\xArgs
}

% 学籍番号
\pfnode[right, text width=5.5cm, text centered] (14.3, 27.0) {\py{mojimoji.han_to_zen(pinfo["学籍番号"])}}

% 氏名
\pfnode[right, text width=6.7cm, text centered] (3.95, 25.8) {\py{pinfo["姓"]} \py{pinfo["名"]}}

% 研究科名
\pfnode[right, text width=5.5cm, text centered] (14.3, 25.8) {工学研究科\\\py{pinfo["所属"]}}

\fontsize{11pt}{6.45mm}\selectfont
\parindent=11pt
% 推薦理由
\pfnode[below right, text width=17.9cm, align=justify] (1.7, 25.05) {\ReasonForRecommendation}

}

\end{document}